\selectlanguage{danish}

\SetLabelAlign{parright}{\parbox[t]{\labelwidth}{\raggedleft#1}}

\begin{description}[style=multiline,nosep,itemsep=4ex,leftmargin=1.5cm,align=parright,labelsep=0.3cm]
	\item[$\bullet$]
	viser ordelementerne i sammensatte ord:
	
	\smallskip

	\hspace*{0.7cm}\begin{tabular*}{0.7\textwidth}{@{}ll@{}}
		\eo{bon\DEL ven\DEL ig\DEL i} & byde velkommen
	\end{tabular*}

	\item[\T]
	 viser, at opslagsordet gentages:

	\begin{itemize}[leftmargin=4cm,labelwidth=3cm, align=left, font=\itshape, itemsep=2ex]
		\item[uforandret] \eo{dek\DEL e}\\
							\rule{0pt}{1.5ex}\eo{post \T~da sekundoj}

		\item[med tilf{\o}jelse af endelse] \eo{danc\DEL ad\DEL o}\\
											\rule{0pt}{1.5ex}\eo{lerni \T{}n} (dancadon)
		
		\item[med udskiftning af endelse] \eo{danc\DEL i}\\
												\rule{0pt}{1.5ex}\eo{la kuloj \T{}as} (dancas)
		
		\item[i et sammensat ord] \eo{baskul\DEL o}\\
									\rule{0pt}{1.5ex}\eo{\T{}ponto} (baskulponto/baskuloponto)

	\end{itemize}

	\item[=] viser et \textit{synonym}, dvs et ord der stort set har samme betydning
	
	\item[$\rightarrow $] betyder \textit{i stedet anbefales}
	
	\item[,] s{\ae}ttes mellem n{\ae}sten ensbetydende overs{\ae}ttelser af
	opslagsordet
	

	\item[;] s{\ae}ttes mellem mere forskellige overs{\ae}ttelser
	
	\item[ogs] 
	viser, at ordet har andre overs{\ae}ttelser end den n{\ae}vnte.
	De andre udledes af grund\-ord\-ets eller et parallelt
	\eo{-ig-} eller
	\eo{{}-i\^g-}ords overs{\ae}ttelser. Den efter \textit{ogs}
	viste overs{\ae}ttelse kan enten v{\ae}re \'en, der ikke kan udledes af
	grundordets overs{\ae}ttelser, eller \'en som brugeren af andre grunde
	b{\o}r g{\o}res opm{\ae}rksom p{\aa}
	
	\item[dgl] Daglig tale er ment som en anbefaling
	af det p{\aa}g{\ae}ldende ord til normal sprogbrug uden for
	specialisternes kreds. I mods{\ae}tning til de esperantoord, der er
	dannet ud fra de noget sterile latinske betegnelser, rummer de her
	anbefalede langt mere af dagligdagens saft og kraft, og de er derfor
	lettere at huske. Sammenlign fx ordene for g{\aa}sefod og for
	hundestejle:

	\smallskip

	\hspace*{0.7cm}\begin{tabular*}{0.7\textwidth}{@{}lll@{}}
	\eo{kenopodi\DEL o} & \eo{anser\DEL pied\DEL o} & g{\aa}sefod\\
	\rule{0pt}{3ex}\eo{gasteroste\DEL o}& \eo{dorn\DEL fi\^s\DEL o}& hundestejle \\
	\end{tabular*}
	
	\medskip
	
	Visse dgl-navne m{\aa} fx g{\o}res til genstand for s{\ae}rlig omhu,
	idet de tilsvarende nationale navne d{\ae}kker forskellige arter i
	forskellige sprog.

\end{description}

\newpage

\section{R{\ae}kkef{\o}lge}

Opslagsordene, herunder sammensatte ord, er ordnet alfabetisk.
Flerordsudtryk er opf{\o}rt under det f{\o}rste ord, medmindre dette er
\eo{la} eller et andet sm{\aa}ord.

\section{Uddybninger}

Uddybninger st{\aa}r i parentes. Hvis der er et komma foran parentesen,
g{\ae}lder den alle foreg{\aa}ende ord eller udtryk med komma imellem,
fx \textbf{an\^co} blad, tunge, (i bl{\ae}seinstrument). Hvis der ikke
er komma foran parentesen, g{\ae}lder den kun det ord eller udtryk der
st{\aa}r lige foran parentesen, fx \textbf{ajn} som helst, ligegyldigt
(hvem, hvilken).

I tilf{\ae}lde hvor der ikke findes en generelt anvendelig dansk
overs{\ae}ttelse, gives en forklaring i parentes lige efter
opslagsordet; en overs{\ae}ttelse m{\aa} s{\aa} i hvert enkelt
tilf{\ae}lde formuleres p{\aa} basis af denne forklaring og eventuelle
eksempler.


\section{Transitivitet}
Verber der \textit{kan} -- men for manges vedkommende ikke n{\o}dvendigvis \textit{skal} -- have genstandsled, kaldes transitive \emph{(tr)}. Det g{\ae}lder fx for \eo{man\^gi}:

\hspace*{0.7cm}\begin{tabular*}{0.7\textwidth}{@{}ll@{}}
\eo{Mi man\^gis bananon} (med genstandsled) & Jeg spiste en banan \\
\rule{0pt}{3ex}\eo{Li man\^gis kiam ni venis} (uden genstandsled) & Han spiste da vi kom. \\
\end{tabular*}

Verber der ikke kan have genstandsled, kaldes intransitive \emph{(itr)}.
Ogs{\aa} verber, der kun kan have genstandsled i meget begr{\ae}nset omfang i s{\ae}rlige tilf{\ae}lde, er intransitive. Et
s{\aa}dant s{\ae}rligt tilf{\ae}lde kan fx v{\ae}re at genstandsleddet og det p{\aa}g{\ae}ldende verbum best{\aa}r af
samme ord i forskellige former. S{\aa}ledes kan \eo{dormi} og \eo{vivi} normalt ikke have genstandsled, men man kan godt sige:
\eo{Ili vivis feli\^can vivon}. De levede et lykkeligt liv. 
Verberne markeres i ordbogen kun med \emph{tr} eller \emph{itr}, hvor dette sk{\o}nnes at v{\ae}re til praktisk nytte.
For det f{\o}rste er alle verber med \eo{-ig-} og \eo{-i\^g-} henholdsvis transitive og intransitive, og
for det andet ses den praktiske sprogbrug ofte af de anf{\o}rte eksempler.